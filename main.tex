\documentclass[12pt]{article} % Definiert die Dokumentenklasse als Artikel mit Schriftgröße 12pt
\usepackage{build/customizations} % Lädt Anpassungen aus einer benutzerdefinierten Datei
\usepackage{build/commands} % Lädt benutzerdefinierte Befehle aus einer Datei


% Einstellungen (EDIT)
\title{TITLE} % Titel des Dokuments
\authorname{FIRSTNAME}{MIDDLENAME}{LASTNAME} % Autorname; den Mittelnamen leer lassen, wenn keiner vorhanden ist
\date{12.07.2024} % Datum der Einreichung oder Veröffentlichung im Format TT.MM.JJJJ
\def\semgroup{SEMGROUP} % Definiert die Seminargruppe
\def\matnum{matnum} % Definiert die Matrikelnummer
\def\pubUniversity{Bautzen, Berufsakademie Sachsen, Staatliche Studienakademie Bautzen} % Name der Universität
\def\papertype{TypeOfPaper} % Art der Arbeit (z.B. Hausarbeit, Bachelorarbeit, etc.)
\def\reviewer{ReviewerName, Reviewer Company?} % Name und Zugehörigkeit des Gutachters
\def\location{Bautzen} % Ort der Unterschrift für die Unabhängigkeitserklärung


\begin{document}

\customtitlepage{build/images/logo.png}{0.1} % Erzeugt eine benutzerdefinierte Titelseite mit einem Logo (Größe 0.1)
\thispagestyle{empty} % Entfernt die Seitenzahl für diese Seite
\customauthorsabstract\par
 Your abstract here. % Abstract hier einfügen


\newpage 

\tableofcontents % Generiert das Inhaltsverzeichnis
\thispagestyle{empty} % Entfernt die Seitenzahl für diese Seite
\newpage

%Entfenen wenn nicht benutzt!
\printglossary[type=\acronymtype, title=Abkürzungsverzeichnis]
\thispagestyle{empty} % Entfernt die Seitenzahl für diese Seite
\newpage

\thispagestyle{empty} % Entfernt die Seitenzahl für diese Seite
\listoftables % Generiert das Tabellenverzeichnis
\newpage

\thispagestyle{empty} % Entfernt die Seitenzahl für diese Seite
\listoffigures % Generiert das Abbildungsverzeichnis
\newpage

\section{Einleitung} % Beginn des ersten Abschnitts
\subfile{sections/introduction} % Fügt die Datei "introduction" aus dem Ordner "sections" ein

\section{Hauptteil} % Beginn des Hauptteils
\lipsum[2] % Beispieltext

\subsection{Hauptteil Unterpunk} %Ein Unterpunkt
\lipsum[2]

\section{Fazit} % Beginn des Fazits
\lipsum[3] % Beispieltext

\newpage
\printbibliography[notkeyword=law, title={Quellenverzeichnis}] % Druckt das Quellenverzeichnis ohne die Kategorie "law"
\addcontentsline{toc}{section}{Quellenverzeichnis} % Fügt das Quellenverzeichnis zum Inhaltsverzeichnis hinzu

\newpage
\printbibliography[keyword=law, title={Rechtsquellenverzeichnis}] % Druckt das Rechtsquellenverzeichnis
\addcontentsline{toc}{section}{Rechtsquellenverzeichnis} % Fügt das Rechtsquellenverzeichnis zum Inhaltsverzeichnis hinzu



\newpage


%ListOfAttachments
\listofattachments % Generiert das Anlagenverzeichnis

\newpage

% Beispiel für Anlagen
\begin{attachment}[Titel Anlage1] % Beginnt eine neue Anlage mit dem Titel "Anlage1"
    \lipsum[4] % Beispieltext für Anlage 1
\end{attachment}

\begin{attachment}[Titel Anlage2] % Beginnt eine neue Anlage mit dem Titel "Anlage2"
    \begin{customlistof}{0cm}{} %formatiert listen wie listoffigures/listoftables wie in den richtlinien
        \lipsum[5] % Beispieltext für Anlage 2
    \end{customlistof}
\end{attachment}

\thispagestyle{empty} % Entfernt die Seitenzahl für diese Seite
\section*{Selbstständigkeitserklärung} % Fügt eine nicht nummerierte Sektion für die Selbstständigkeitserklärung hinzu
\subfile{build/sections/declaration of independence} % Fügt die Datei "declaration of independence" aus dem Ordner "sections" ein

\end{document} % Ende des Dokuments
