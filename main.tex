\documentclass[12pt]{article}

\usepackage[utf8]{inputenc}
\usepackage[german]{babel}

%\userpackage{pdfpages}
\usepackage{lipsum}
\usepackage{amsmath}
\usepackage{hyperref}
\usepackage{fancyhdr}
\usepackage{acronym}
\usepackage{amsfonts}
\usepackage{amssymb}
\usepackage{graphicx}
\usepackage{xcolor}
\usepackage{titling}
\usepackage{listings}
\usepackage{float}
\usepackage{lastpage}
\usepackage{setspace}
\usepackage{csquotes}
\usepackage{pdfpages}
\usepackage{tikz}
\usepackage[titles]{tocloft}

\usepackage{helvet}

\graphicspath{ {build/images}{images/} }
%DEF

\def\authorname#1#2#3{%
\def\firstname{#1}%
\def\middlename{#2}%
\def\lastname{#3}%
\gdef\theauthor{#1\ #2\ #3} % Globale Definition für den Gebrauch in \author
\gdef\reversedauthor{#3, #1\ #2} }

%BIBLATEX

\usepackage[
	backend=biber,
	%style=alphabetic,
]{biblatex-chicago}

\setlength{\bibitemsep}{1em} % Abstand zwischen den Literaturangaben
\setlength{\bibhang}{2em} % Einzug nach jeweils erster Zeile

% Bibliography file
\addbibresource{build/bib/law.bib}
\addbibresource{build/bib/references.bib}

\DeclareBibliographyCategory{law}
\addtocategory{law}{aktg1965}

\DeclareFieldFormat[article,book,inbook,incollection,inproceedings,misc,thesis,unpublished]{title}{#1}

\renewbibmacro{in:}{%
\ifentrytype{article}{% Nur bei Artikeln
\printtext{\bibstring{in}\intitlepunct}}{}}

\DeclareBibliographyDriver{misc}{%
\usebibmacro{bibindex}%
\usebibmacro{begentry}%
\usebibmacro{author/editor+others/translator+others}%
\setunit{\printdelim{nametitledelim}}\newblock \usebibmacro{title}%
\setunit{\space}\newblock \printfield{journaltitle}%
\setunit*{\addspace}\newblock \printfield{volume}%
\setunit*{\addcomma\space}\newblock \printfield{pages}%
\setunit{\addcomma\space}\newblock \printfield{note}%
\usebibmacro{finentry}}

\linespread{1.5}
\setlength{\parindent}{0em}

%Page Numbering
\pagestyle{fancy}

\fancyhf{}
\renewcommand{\headrulewidth}{0pt} % remove the header rule
\fancyhead[C]{\thepage} % page number in the center of the header

%Verzeichnisformatierung
\renewcommand{\cftdot}{}

\renewcommand{\cftfigdotsep}{\cftnodots}
\renewcommand{\cftfigpresnum}{Abbildung }
\renewcommand{\cftfigaftersnum}{:}
\setlength{\cftfignumwidth}{6em}

\renewcommand{\cftfigdotsep}{\cftnodots}
\renewcommand{\cftfigpresnum}{Tabelle }
\renewcommand{\cftfigaftersnum}{:}
\setlength{\cftfignumwidth}{6em}

%GLOSSARIES

\usepackage{glossaries}
\newglossary[slg]{acronyms}{syi}{syo}{Symbol List}
\loadglsentries{build/acronyms/acronyms.tex}
\makeglossaries

%Formationseinsteillung

\usepackage[a4paper, left=40mm, right=20mm, top=20mm, bottom=20mm,]{geometry}


%Settings
%TODO: Anpassen
\title{[Thema]}
\authorname{Hans}{Peter}{Musterstudent} %First Name, Middle Name, Last name
\date{xx.xx.xxxx} %Publish Date
\def\semgroup{1BRXX-1} %Seminar Group
\def\matnum{100XXXX} %matriculation number
\def\pubUniversity{Bautzen, Berufsakademie Sachsen, Staatliche Studienakademie Bauzen} % University
\def\papertype{Reflexionsbericht/Seminararbeit} % type of paper
\def\reviewer{Titel Max Musterbetreuer, Firma} % reviewer


%Setauthor
\author{\theauthor}


\usepackage{subfiles}

\begin{document}
	\begin{titlepage}
            {\fontfamily{phv}\selectfont
            \includepdf[pagecommand={\begin{tikzpicture}[remember picture, overlay]
		\node at ([xshift=0.2cm,yshift=6.68cm] current page.center) {{\centering \textbf{\thetitle}}};
            \node at ([xshift=.188cm,yshift=2.36cm] current page.center) {{\centering \papertype}};
            \node at ([xshift=0.188cm,yshift=-2.53cm] current page.center) {{\centering \theauthor}};
            \node at ([xshift=0.188cm,yshift=-3.07cm] current page.center) {{\centering Seminargruppe \semgroup}};
            \node at ([xshift=0.188cm,yshift=-3.51cm] current page.center) {{\centering Matrikelnr. \matnum}};
            \node at ([xshift=0.188cm,yshift=-5.0cm] current page.center) {{\centering am \thedate}};
            \node[anchor=west] at ([xshift=-8.27cm,yshift=-9.36cm] current page.center) {\reviewer};
            \node at ([xshift=-6.3cm,yshift=12.4cm] current page.center) {\includegraphics[scale=0.2] {build/images/logo.png}};
	\end{tikzpicture}}]{build/images/title.pdf} %TODO Anpassen!!
            }
        \end{titlepage}

	\thispagestyle{empty}
	\section*{Autorenreferat}
	\reversedauthor:
	\newline
	\thetitle{\ -} \the\year{} - \pageref{LastPage} Bl.
	\newline
	\pubUniversity, \papertype, \the\year{}
	\newline
	\par Your abstract here.
	%\thispagestyle{empty}

	\newpage

	%\section*{Inhaltsverzeichnis}
	\tableofcontents
	\thispagestyle{empty}
	\newpage

	%Verzeichnisse opt.
	%\section*{Abkürzungsverzeichnis}
	\thispagestyle{empty}
	\printglossary[type=\acronymtype, title=Abkürzungsverzeichnis]
	\newpage

	\section{Einleitung}
	\subfile{sections/introduction}

	\section{Hauptteil}
	\lipsum[2]

	\section{Fazit}
	\lipsum[3]
        

	\newpage
	%\section*{Quellenverzeichnis}
	\printbibliography
	[notcategory=law, title=Quellenverzeichnisverzeichnis, heading=bibnumbered]
	\newpage

	%\section*{Rechtsquellenverzeichnis}
	\printbibliography
	[category=law, title=Rechtsquellenverzeichnis, heading=bibnumbered]
	\newpage

	\section{Anhang}
	Your attachments here.

	\subsection{Anlagenverzeichnis}
	Your attachments here.

	\newpage
	\listoffigures
	\newpage

	\newpage
	\listoftables
	\newpage

	%\makeglossaries

	\newpage
	\thispagestyle{empty}
	\section*{Selbstständigkeitserklärung}

	Ich (Wir) erkläre(n) an Eides statt, dass ich (wir) die vorliegende Arbeit (entsprechend
	der genannten Verantwortlichkeit) selbstständig und nur unter Verwendung der angegebenen
	Quellen und Hilfsmittel angefertigt habe(n).
	\medskip

	Die Zustimmung der Firma zur Verwendung betrieblicher Unterlagen habe(n) ich (wir)
	eingeholt. Die Arbeit wurde bisher in gleicher oder ähnlicher Form weder
	veröffentlicht noch einer anderen Prüfungsbehörde vorgelegt.
	\medskip

	Ich versichere weiterhin, dass die auf elektronischem Wege eingereichten Unterlagen
	mit den schriftlichen Ausfertigungen übereinstimmen.

	\vspace{15mm}
	\hfill%
	\begin{tabular}[t]{c}
		\rule{10em}{0.4pt}      \\
		Theodor Ferdinand Frank
	\end{tabular}%
	\hfill%
	\begin{tabular}[t]{c}
		\rule{10em}{0.4pt} \\
		Dresden, \today
	\end{tabular}%
	\hfill\strut
\end{document}